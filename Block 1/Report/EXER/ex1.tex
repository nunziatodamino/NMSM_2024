\chapter*{Lecture 1: Crude Monte Carlo and inversion sampling}

\subsection*{Sampling random points within D-dimensional domains by hit and miss}

\begin{wrapfigure}{r}{0.5\textwidth}
    \vspace{-25pt}
    \centering
    \includegraphics[width=0.33\textwidth]{FIG/exercise_0_images/first_ellipsoid_distribution.png}
  \caption{Volume distribution of the Monte Carlo estimate by hit and miss}
  \label{lec1:first_ellipsoid}
  \end{wrapfigure}

We want to evaluate the volume of an ellipsoid of given semi-axes $a=3,\ b=2,\ c=2$. 
In order to reduce the error on our estimate from the get go, we make the following consideration. We can approach the problem in 2 ways : 
we can consider an integration box containing the whole ellipsoid and proceed with our calculation, or use to our advantage the problem's symmetry, 
and use as integration box the one containing one octant of the ellipsoid, and then multiply our estimate by 8. \\
The latter choice has the effect to reduce the variance in our sampling, and by consequence the error on our estimate. \\
This happens because , given that each hit or miss procedure is in fact a Bernoulli trial, the error on the volume estimate is evaluated by :

$$ \sigma_V = \sqrt{\frac{p( 1-p )}{N}} V_{\text{integration box}} $$

Where $p$ is our success probability, i.e. the probability to hit inside the integration box, and $N$ is the number of trials, i.e. the number of iterations.\\
So, as we seen $\sigma_V \propto V_{\text{integration box}}$, so the smaller it is, better is our estimate. \\
Given this consideration we report in figure \ref{lec1:first_ellipsoid} the result obtained using the procedure with the ellipsoid octant for an ellipsoid of given semi-axes $a=3,\ b=1,\ c=1$:

$$\langle V \rangle_1 \simeq 6.283 \pm 0.060$$

Now we propose to evaluate the volume of an ellipsoid of given semi-axes $a=3,\ b=1,\ c=1$. \\
From the previous considerations we'll expect to have on this evaluation a smaller error, specifically, considering a situation in which $p$
and $N$ are the same, we can expect an error that is $\frac{V_{\text{box1}}}{V_{\text{box2}}} = 4$ times smaller, even before running any simulation. \\
We obtain for this ellipsoid:

$$\langle V \rangle_2 \simeq 1.571 \pm 0.015$$

\begin{wrapfigure}{H}{0.5\textwidth}
  \vspace{-20pt}
  \centering
  \includegraphics[width=0.33\textwidth]{FIG/exercise_0_images/first_and_second_ellipsoid_distribution.png}
\caption{Volume distribution comparison between the 2 ellipsoids}
\label{lec1:first_and_second_ellipsoid}
\end{wrapfigure}

Which confirms our prediction on the error. One can see also graphically this by comparing the 2 volume distributions and see in figure \ref{lec1:first_and_second_ellipsoid} that the one with the
smaller integration box is narrower. \\
Moreover, given that the same volume can be evaluate analitically, we can study the deviation of the estimate from the analytical value. \\
In figure \ref{lec1:deviation_comparison} we report for each of the 2 simulations the deviation from the analytical value, i.e $\Delta f = \mid f_{th}-\langle f \rangle \mid $ as a function
of the number of iterations.

\begin{figure}[h]
  \centering
  \includegraphics[width=0.7\textwidth]{FIG/exercise_0_images/relative_error_trend.png}
  \caption{Comparison between the relative deviations for the 2 ellipsoids and the expected theroretical trend}
  \label{lec1:deviation_comparison}
  \end{figure}


Notice that, as discussed before, even if having a bigger number of iterations has a net effect of diminishing the error, the bigger variance in 
the case of the first ellipsoid makes the deviations bigger respect to the second ellipsoid. \\
We plotted for reference the general trend $\frac{1}{\sqrt{N}}$ which is the one to be expected in this kind of curves.  


\subsection*{Sampling random numbers from a given distribution: inversion method}

We start from the following PDF's:

$$ \rho_1(x) = cxe^{-x^2}\ ,x \in \mathbb{R}^+ \qquad \rho_2(x) = bx^4\ , x \in [0,3]$$

We first proceed by doing the normalization to find the coefficients $c$ and $b$:

$$ \int_{\mathbb{R}^+}\rho_1(x)dx =1 \iff c=2 \qquad \int_{0}^3\rho_2(x)dx =1 \iff b = \frac{5}{243} $$

Then one can evaluate the CDF's by definition:

$$ F_1(t)=\int_{0}^t\rho_1(x)dx = 1-e^{-t^2}, \ t \in \mathbb{R}^+ \qquad F_2(t)=\int_{0}^t\rho_2(x)dx = \frac{t^5}{243}, \ t \in [0,3] $$

Finally by inversion:

$$ F^{-1}_1(y) = \sqrt{\ln\left(\frac{1}{1-y}\right)} \qquad F^{-1}_2(y)= \sqrt[5]{243 y}$$

For both $y \in (0,1]$.

In order to implement this we use algorithm \ref{inversion}. We report in the following the obtained graphs:

\begin{figure}[H]
\centering
\includegraphics[width=0.48\textwidth]{FIG/exercise_1_images/Figure_1.png}
\hfill
\includegraphics[width=0.48\textwidth]{FIG/exercise_1_images/Figure_2.png}
\caption{Sampling from distribution $\rho_1$ and $\rho_2$ using the inversion method.}
\label{fig:combined_figure}
\end{figure}