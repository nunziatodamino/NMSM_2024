\chapter*{Appendix : algorithms}

\begin{algorithm}
    \caption{Inversion Sampling Algorithm}
    \begin{algorithmic}[1]
    \State \( S \leftarrow \emptyset \)
    \State \( \text{count} \leftarrow 0 \)
    \While{count < N}
        \State Sample \( U \sim \text{Uniform}(0, 1) \) \Comment{Generate a uniform random number}
        \State Compute \( X \leftarrow F^{-1}(U) \) \Comment{Apply the inverse CDF to transform \( U \)}
        \State Append \( X \) to \( S \)
        \State \( \text{count} \leftarrow \text{count} + 1 \)
    \EndWhile
    \State \textbf{Return} \( S \)
    \end{algorithmic}
    \label{inversion}
\end{algorithm}


\begin{algorithm}
    \caption{Rejection Sampling Algorithm}
    \begin{algorithmic}[1]
    \State \( S \leftarrow \emptyset \)
    \State \( \text{count} \leftarrow 0 \)
    \While{count < N}
        \State Sample \( X \sim g(x) \) \Comment{Generate sample from candidate distribution}
        \State Sample \( U \sim \text{Uniform}(0, 1) \) 
        \State \( A \leftarrow \frac{f(X)}{c \cdot g(X)} \) \Comment{Calculate acceptance threshold}
        \If{$U \leq A$}
            \State Append \( X \) to \( S \)
            \State \( \text{count} \leftarrow \text{count} + 1 \)
        \EndIf
    \EndWhile
    \State \textbf{Return} \( S \)
    \end{algorithmic}
    \label{rejection}
    \end{algorithm}
    
\phantomsection
\label{par:detail_rejection}  
Implementation detail : Note that to obtain exactly $N$ samples a while loop must be used in the algorithm. \\
Moreover is interesting to discuss some technical details regarding the implementation of this algorithm in Python when the candidate
distribution is a truncated distribution, or in general a distribution that cannot be generated easily but needs the \fbox{\texttt{scipy.stats}} package
to be generated.\\
In this case a naive implementation of algorithm \ref{rejection} can hinder the perfomance greatly, because even if one has in mind to speed
up the function with a \fbox{\texttt{numba}} decorator, this will not work, due to known incompatibility issues within the 2 cited packages. \\
The easiest solution to this problem is to vectorize our algorithm using the \fbox{\texttt{numpy}} library, overproducing samples 
and then cutting what is not necessary. This approach has a clear caveat, we risk an \fbox{\texttt{int}} overflow, so we must put attention
in not choosing an high value of $N$ for this solution. If a big number of samples is required, this approach must be revised.

  \begin{algorithm}
    \caption{Importance Sampling Algorithm}
    \begin{algorithmic}[1]
    \State \( \mu \leftarrow 0 \) 
    \State \( \text{count} \leftarrow 0 \)
    \While{count < N}
        \State Sample \( X \sim g(x) \) \Comment{Generate sample from proposal distribution \( g(x) \)}
        \State \( \mu \leftarrow \mu + \frac{f(X)}{g(X)}\rho(X) \)
        \State \( \text{count} \leftarrow \text{count} + 1 \)
    \EndWhile
    \State \textbf{Return} \( \mu / N \) 
    \end{algorithmic}
    \label{importance_sampling}
\end{algorithm}