\chapter*{Lecture 7: Brussellator}

The Brussellator is a model that describes an autocatalytic and oscillatory chemical reaction. In literature is given as follows:

\begin{equation}
    \begin{cases} 
        A \xrightarrow{K_1} X, & \text{(1a)} \\ 
        B + X \xrightarrow{K_2} Y + D, & \text{(2a)} \\
         2X + Y \xrightarrow{K_3} 3X, & \text{(3a)} \\ 
         X \xrightarrow{K_4} E. & \text{(4a)}     
    \end{cases}    
\label{bruss_orig}
\end{equation}

Notice that for the reaction dynamics, only the third equation is important, i.e. the autocatalysis reaction $2X + Y \rightarrow 3X$, while
the other equations describe ,in order, the $X$ production (1a), the $X \to Y$ conversion (2a), the $Y$ depletion (4a). \\
So, in a context in which we are interest in simulating just the autocatalysis dynamics and not the system details, we can reduce 
the set of reactions to the following one:

$$
\begin{cases}
    \emptyset\xrightarrow{k_1}A_X \\
    A_X\xrightarrow{k_2}\emptyset  \\
    2A_X+A_Y \xrightarrow{k_3} 3A_X\\
    A_X\xrightarrow{k_4} A_Y
\end{cases}
$$

Where with $A_X,A_Y$ we denote molecules of two different species. \\ The only hypothesis one has to make is that the molecules $A$ and $B$ are
in high concentration so that the reactions (1a) and (2a) can happen with no problem. \\
This implies automatically that our system will perform limit cycles and the stationary state will be unstable and oscillatory.\\
This because, from the stability analysis of \ref{bruss_orig}, we have that the system is unstable if the following condition is satisfied:

$$ K_2 [B] - K_4 - K_3[X_{eq}]^2 > 0 $$

Which is satisfied for an high $B$ concentration. \\

Instead of going through the deterministic route, we can consider the stochastic version of these reactions, transforming in this way the deterministic dynamics into a Markov process. \\

We denote from now with $ \CC=(X,Y)$ a state of the system where $X$ and $Y$ represents the number molecules of the two species $A_X$ and $A_Y$
respectively. The transition rates for a state $ \CC=(X,Y)$ in this process are given by the equations:

$$
\begin{cases}
    w_{1} = a\Omega \quad &\text{for}\quad(X,Y)\to(X+1,Y)\\
    w_{2} = X\quad&\text{for}\quad(X,Y)\to(X-1,Y)\\
    w_{3} = \frac{1}{\Omega^{2}}X(X-1)Y\quad&\text{for}\quad(X,Y)\to(X+1,Y-1)\\
    w_{4} = bX\quad&\text{for}\quad(X,Y)\to(X-1,Y+1)
\end{cases}
$$

Notice that these equations introduces the finite size of the system via a volume $\Omega$. 
We study in the following the effects on evolution for different volumes ($\Omega = 10^2, 10^3, 10^4$) , given $a = 2$, $b = 5$.  \\
We report in figure \ref{lec7:moleculesxy_comparison} the molecules orbits for the different volumes.

\begin{figure}[t!]
    \centering
    \includegraphics[width=0.33\textwidth]{FIG/exercise_7_images/gillespiexy_volume100.0.png} \hspace{0.5cm}
    \includegraphics[width=0.33\textwidth]{FIG/exercise_7_images/gillespiexy_volume1000.0.png} \hspace{0.5cm}
    \includegraphics[width=0.33\textwidth]{FIG/exercise_7_images/gillespiexy_volume10000.0.png}
    \caption{Comparison between the molecules graphs for different volumes}
    \label{lec7:moleculesxy_comparison}
\end{figure}

As expected we can see we have a limit cycle for each volume, and moreover we expect an oscillatory, described by picture \ref{lec7:moleculesxyevolution_comparison}.

\begin{figure}[H]
    \centering
    \includegraphics[width=0.6\textwidth]{FIG/exercise_7_images/gillespiexy_time_volume100.0.png} 
    \vfill
    \includegraphics[width=0.6\textwidth]{FIG/exercise_7_images/gillespiexy_time_volume1000.0.png} 
    \vfill
    \includegraphics[width=0.6\textwidth]{FIG/exercise_7_images/gillespiexy_time_volume10000.0.png}
    \caption{Comparison between the molecules graphs for different volumes}
    \label{lec7:moleculesxyevolution_comparison}
\end{figure}

Notice that the volume has a regularizing effect on the oscillations.