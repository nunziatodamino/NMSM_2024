\chapter*{Lecture 4: Markov chains}

\paragraph{Part 1} We report the stochastic matrixes with the relative digraphs in figure \ref{chapter4:part1}. \\
One can prove from the digraph representation that chain $A$ is irreducible, while chain $B$ is not (for example in chain $B$ from state 2 
is not possible to reach any other state).

\begin{figure}[H]
    \centering
    \includegraphics[width=0.8\textwidth]{FIG/exercise_4_images/Exercise 4_part1.png}
    \caption{Stochastic matrixes with relative digraph representation}
    \label{chapter4:part1}
\end{figure}

Moreover on chain A each state is periodic of period 3. In chain B state 1 is aperiodic and transient, state 2 is absorbing and states 3,4,5 are periodic of period 3.

\paragraph*{Part 2}

Suppose to have the Markov chain with the following stochastic matrix:

$$ P =
\begin{pmatrix} 
\frac{1}{2} & \frac{1}{3} & \frac{1}{6} \\
\frac{3}{4} & 0 & \frac{1}{4} \\
0 & 1 & 0 \\
\end{pmatrix}
$$
One can show with the corresponding digraph in figure \ref{chapter4:part2} that is irreducible. \\

\begin{figure}[H]
    \centering
    \includegraphics[width=0.5\textwidth]{FIG/exercise_4_images/Exercise 4_part2.png}
    \caption{Stochastic matrixes with relative digraph representation}
    \label{chapter4:part2}
\end{figure}

Moreover, given that state 1 is aperiodic, we can imply that the whole chain is aperiodic. \\

Now suppose to start the process in state 1, and that we want to evaluate the probability that it is in state 3 after two steps. \\
This can be easily accomplished by evaluating $P^2$ and then selecting from this matrix the element $P^2_{13} = \frac{1}{6}$.

Now we evaluate the stochastic matrix in the limit $n \to +\infty$ in 2 ways, first by using the analytical approach and the by computation trying to estimate it.
First we evaluate the invariant distribution:

$$ \bar{\pi}P = \bar{\pi} \iff P^t\bar{\pi}^t = \bar{\pi}^t $$

In the end one as to solve this linear system with the normalization constraint $\sum_i\pi_i=1$.
We then obtain the following linear system:

$$
\begin{cases}
\frac{1}{2}\pi_1 + \frac{3}{4}\pi_2 = \pi_1 \\
\frac{1}{3}\pi_1 + \pi_3 = \pi_2 \\
\frac{1}{6}\pi_1 + \frac{1}{4}\pi_2 = \pi_3 \\
\pi_1 + \pi_2 + \pi_3 = 1
\end{cases} \iff \bar{\pi} = \left( \frac{1}{2}, \frac{1}{3}, \frac{1}{6} \right)
$$

The limiting stochastic matrix will have for each row the invariant distribution:

$$ P^{\infty} =
\begin{pmatrix} 
\frac{1}{2} & \frac{1}{3} & \frac{1}{6} \\
\frac{1}{2} & \frac{1}{3} & \frac{1}{6} \\
\frac{1}{2} & \frac{1}{3} & \frac{1}{6} \\
\end{pmatrix}
$$

We obtain the same result in the submitted program.

\paragraph*{Balls and boxes}

We can solve this exercise using a Markov chain approach.
Called $a_i$ the state in which there are $i$ red balls in box A there can be just 3 possible states (0,1,2 balls).
So we need to evaluate the transition probabilities. In general, for a given state $i$:

\begin{itemize}
\item  $a_i \to a_{i+1}$ = $\mathbb{P}$(red extraction in B) and $\mathbb{P}$(white extraction in A)
\item  $a_i \to a_{i-1}$ = $\mathbb{P}$(red extraction in A) and $\mathbb{P}$(white extraction in B)
\item  $a_i \to a_{i}$ = ($\mathbb{P}$(red extraction in B) and $\mathbb{P}$(red extraction in A)) or ($\mathbb{P}$(white extraction in B) and $\mathbb{P}$(white extraction in A))
\end{itemize}

Now we can proceed to evaluate these probabilities and then evaluate the transition matrix elements.
In general one has, for a given state $i$:
$$ \mathbb{P}\text{(red extraction in A at state i)} = \frac{i}{2} \qquad \mathbb{P}\text{(white extraction in A at state i)} = 1- \frac{i}{2} = \frac{2-i}{2} $$
$$ \mathbb{P}\text{(red extraction in B at state i)} = 1 - \frac{i}{3} =  \frac{3-i}{3} \qquad \mathbb{P}\text{(white extraction in B at state i)} = \frac{i}{3} $$

We obtain the stochastic matrix

$$ P =
\begin{pmatrix} 
0 & 1 & 0 \\
\frac{1}{6} & \frac{1}{2} & \frac{1}{3} \\
0 & \frac{2}{3} & \frac{1}{3}
\end{pmatrix}
$$

Which is a stochastic matrix because all rows sums up to 1. \\
Now the ansatz is the same as before. To evaluate the occurrence probability of state $a_2$ after 3 events we consider the element $P^3_{13}$ (note that our indexes are shifted by 1). \\
We obtain $P^3_{13} \simeq 0.28$, while in the limit of many events $P^{\infty}_{13} = 0.3$