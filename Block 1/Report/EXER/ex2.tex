\chapter*{Lecture 2: Rejection method}

The rejection sampling method is useful when our pdf is not easily invertible, and so the inversion method cannot be applied. \\
Suppose to have as pdf:

$$ \rho(x) = A_f \exp\left( -8 \left( \frac{x^2}{2}+\frac{x^4}{4} \right) \right) $$

Imposing the normalization condition we obtain:

$$ \int_{\mathbb{R}} \rho(x) dx = 1 \qquad \iff \qquad   \frac{eA_f}{\sqrt{2}}K_{1/4}(1) = 1$$

Where we denote with $K_{1/4}(x)$ the Bessel function of the second kind. 
We find that $A_f \simeq 1.21$.
Now we want to find an invertible pdf $g(x)$ (that we'll call candidate density) with the property:

$$ \rho(x) \leq c g(x) \qquad \forall x \in S $$

Where $S$ is the sample space of $\rho(x)$.
The closer the function $cg(x)$ is to $\rho(x)$ the more efficient the sampling is. \\

\begin{wrapfigure}{r}{0.5\textwidth}
    \vspace{-25pt}
    \centering
    \includegraphics[width=0.5\textwidth]{FIG/exercise_2_images/rejection_sampling_inequality_check.png}
    \caption{Graphical method for evaluating the inequality $\rho(x)< cg(x)$ in the rejection method}
    \label{chapter2_rej_ineq}
\end{wrapfigure}


There can be different ways to choose the candidate density. We explore different possibilities in 1D and higher dimensions. \\
First we must start with a guess for the candidate density and the $c$ parameter. Then we see if this guess respects the proposed inequality: if it does we are done, if not we do another guess and try again. \\ 
In 1D the easiest way to to test our guess is the graphical one: we plot $\rho(x)$ and $cg(x)$ in the same graph and evaluate if the inequality holds. Another way can be to evaluate point-wise the inequality.
In dimensions where no graphical rapresentation is available, the second method can be easily extended and in order to reduce the computational complexity instead of starting with a proper point-wise evaluation, we can roughly grid the hyper-domain to check if our guess of the candidate density is sensible. If it is, we refine the grid and iterate the process until we are satisfied.

For our exercise, we can choose as candidate density $\mathcal{N}(0,\frac{1}{2})$ with c = 1.6, as shown in the picture \ref{chapter2_rej_ineq} \\

Then the sampling is done with algorithm \ref{rejection}. 
One can show graphically that the sample is good by plotted the properly binned random variables obtained by the process together with the 
target density, as showed in figure \ref{chapter2_rej_variables}

\begin{figure}[t!]
\centering
\includegraphics[width=1\textwidth]{FIG/exercise_2_images/rejection_sampling_verify.png}
\caption{Sampled variables with target density}
\label{chapter2_rej_variables}
\end{figure}
        