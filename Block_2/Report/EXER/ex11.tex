\chapter*{Exercise 11 : Langevin and Brownian dynamics}
\subsection*{Pen \& Paper - Brownian time and length scales}
We begin by considering the Langevin equation in the form:

$$ m\ddot{\bar{x}} + m\gamma\dot{\bar{x}} + \sqrt{6 m \gamma k_BT} \bar{\xi}(t) = 0 $$

In the particular case in which the particle is considered spherical, with a diameter $\sigma = 10^{-8}\ m$, has roughly the same density of water and diffuse in water, we have
in the limit of low Reynolds number that $m\gamma = 3 \pi \eta \sigma$, where $\eta$ is the water viscosity.\\
With the parameters considered we have:

$$ m = \frac{4}{3}\pi \rho\frac{\sigma^3}{8} \sim 10^{-22}\ kg \qquad m\gamma = 3 \pi \eta \sigma \sim 10^{-10}\ kg \cdot s^{-1} $$

So in this case we can neglect the inertial force and write:

$$ m\gamma\dot{\bar{x}} + \sqrt{6 m \gamma k_BT} \bar{\xi}(t) = 0 $$

Implying that we are in the diffusion regime. We can then simply estimate the time to diffuse in water over a distance equal to the particle diameter by :

$$ \sigma ^ 2 = 6Dt = 6 \frac{k_BT}{3\pi\eta\sigma} t \qquad \iff \qquad t \sim 4 \cdot 10^{-7}\ s $$

\subsection*{Numerical exercise} We consider a gas of $N = 1000$ non interacting Brownian particle in a simulation for the reference units are set as $\sigma = 1$, $\varepsilon = 1$ and $m=1$. 
The simulation box is a cubic box of side $L = 20\sigma$, where periodic boundary conditions are considered. We start by setting all the particles in the origin of our simulation box. \\
In order to study the diffusity properties of the system we decided to study it in unwrapped coordinates. \\
\paragraph{Brownian motion in the bulk} In absence of any potential we study the system in the overdamped limit, by implementing for its evolution a first order integrator such as the Euler-Maruyama integrator, which algorithm can be found in the relative section. \\
We study the system mean square displacement for varying temperature values $0.1 < T^* < 2$ while $\gamma^* = 1$, and obtain figure \ref{ex11:msd_temp}.

\begin{figure}[htp]
    \centering
    \includegraphics[width=0.8\textwidth]{FIG/ex11/msd_temp.png}
    \caption{Mean square displacement for $0.1 < T^* < 2$ while $\gamma^* = 1$}
    \label{ex11:msd_temp}
\end{figure}

All MSD are proportional to time as expected and clearly higher the temperature, higher the kinetic energy and consequently the diffusivity. \\
Suppose instead we fix $T^* = 1$ and vary the friction coefficient $10 < \gamma^* < 100 $. Then we obtain \ref{ex11:msd_coeff}

\begin{figure}[htp]
    \centering
    \includegraphics[width=0.8\textwidth]{FIG/ex11/msd_friction.png}
    \caption{Mean square displacement for $10 < \gamma^* < 100$ while $T^* = 1$}
    \label{ex11:msd_coeff}
\end{figure}

Then we obtain the converse, higher the friction coefficient, lower the diffusivity. Finally we can plot the distribution of the $N$ particles along the x axis for
 $T^* = 1$ and $\gamma^* = 1$. We obtain \ref{ex11:x_pos} :

 \begin{figure}[htp]
    \centering
    \includegraphics[width=0.8\textwidth]{FIG/ex11/x_component.png}
    \caption{Distribution of the x position for 1000 Brownian particles}
    \label{ex11:x_pos}
\end{figure}

As expected, due to a non zero diffusivity, particles spreads over time.

\paragraph{Brownian motion in an harmonic trap} Suppose now we set an harmonic trap in the simulation box origin. 
Then if we plot the MSD by varying the harmonic constant $0.1 < K^* < 10$ by letting the other parameter be unitary we obtain figure \ref{ex11:msd_harm}:

\begin{figure}[htp]
    \centering
    \includegraphics[width=0.8\textwidth]{FIG/ex11/mds_hconst.png}
    \caption{Mean square displacement for $0.1 < K^* < 10$ while $T^* = 1$, $\gamma^* = 1$ }
    \label{ex11:msd_harm}
\end{figure}

As we can see the higher the harmonic constant is higher the more pronounced the caging effect around the harmonic trap is. 
This can be seen also by the x position distribution in presence of $K^* = 10$ in figure \ref{ex11:x_pos_harm}

\begin{figure}[htp]
    \centering
    \includegraphics[width=0.8\textwidth]{FIG/ex11/x_component_harm.png}
    \caption{Distribution of the x position for 1000 Brownian particles with $K^* = 10$}
    \label{ex11:x_pos_harm}
\end{figure}

In this case the harmonic trap prevents the natural diffusivity seen in the bulk case, caging the particles.


