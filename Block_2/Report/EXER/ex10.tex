\chapter*{Exercise 10 : Interaction potentials \& thermostats}

\subsection*{Pen \& Paper - Canonical fluctuactions}

We want to prove that, in a canonical ensemble (NVT) : 

$$ \frac{\sigma^2_{T_K}}{\langle T_K \rangle^2} = \frac{\langle T^2_K \rangle- \langle T_K \rangle^2}{\langle T_K \rangle^2}$$

\paragraph{Proof} - From the equipartition theorem, in the NVT ensemble:

$$ \langle T_K \rangle = \frac{2}{3Nk_B} \langle E_K \rangle \qquad \langle T^2_K \rangle = \left(\frac{2}{3Nk_B}\right)^2 \langle E^2_K \rangle $$

So that the temperature variance can be written as :

$$ \sigma^2_{T_K} = \langle T^2_K \rangle-\langle T_K \rangle^2 = \left(\frac{2}{3Nk_B}\right)^2 (\langle E^2_K \rangle - \langle E_K \rangle^2) $$

Moreover we have that the energy variance can be written in the NVT ensemble as:

$$ \sigma^2_{E_K} = \langle E^2_K \rangle - \langle E_K \rangle^2 = k_B T^2 C_V $$

Also the heat capacity :

$$ C_V = \frac{3}{2}k_B T $$

Putting all together:

$$ \sigma^2_{T_K} = \left(\frac{2}{3Nk_B}\right)^2 k_B T^2 \left(\frac{3}{2}k_B T \right) = \frac{2}{3N}T^2 $$

Then one can evaluate the initial statement, in a NVT ensemble :

$$ \frac{\sigma^2_{T_K}}{\langle T_K \rangle^2} = \frac{\frac{2}{3N}T^2}{T^2} = \frac{2}{3N}$$

\subsection*{Numerical exercise}

The idea of the simulation is to simulate a LJ fluid in different ensembles. The simulation will be carried in LAMMPS. \\
For both ensembles we will consider a LJ system with reference units $\sigma=1$, $\varepsilon = 1$ and $m = 1$ in a cubic simulation box of size $L = 10 \sigma$ with periodic boundary conditions. 
The equation of motions will be integrated using the velocity Verlet algorithm, as is the default option in LAMMPS. In each run some steps are discarded for equilibration.

\paragraph{NVE ensemble} We begin by fixing the number density at $\rho = 0.2\sigma^{-3}$ and draw the initial velocities of the system from the Maxwell-Boltzmann distribution at $T^* = 1$, enforcing the total momentum to zero.
We then study the system for different cutoffs, specifically we go from $r_c = 2^{1/6}\sigma$ to $r_c = 4\sigma$ with steps of $\Delta r_c = 0.2\sigma$. \\
For each run we discard 50000 steps for equilibration and consider 150000 steps for production. \\
We observe that at greater cutoff radii the total energy time series stabilize faster and with less variance around the mean value, which is higher for higher cutoff radii. \\
We plot at title of example the energy time series of the cutoff radius $r_c = 2^{1/6}\sigma$ and $r_c = 4\sigma$ after equilibration to show this in figure \ref{ex10:en_comparison}:

\begin{figure}[H]
    \centering
    \includegraphics[width=0.9\textwidth]{FIG/ex10/energy_time_series.png}
    \caption{Energy time series comparison after equilibration for the cutoff radius $r_c = 2^{1/6}\sigma$ and $r_c = 4\sigma$}
    \label{ex10:en_comparison}
\end{figure}

It seems then that one should always use higher cutoff radii, but the problem is that the higher the cutoff radius, the higher the computation time.
We have for example for 200000 iteration for a cutoff radius $r_c = 4\sigma$ a computation time $t_{computation} \sim 8\ s $ while for $r_c = 2^{1/6}\sigma$ a computation time $t_{computation} \sim 1\ s $.\\
Of course the general idea is to find a balance in the choice of a cutoff radius that a guarantees proper stationarity for the observable time series we want to investigate but also has a decent computation time. \\
The rest of the data is not reported, but can be found in the attached log.lammps file. \\
In order to evaluate better the difference between the different cutoff radii, we plot also the radial distribution function for the different cutoffs in figure \ref{ex10:rdf_comparison}. \\

\begin{figure}[H]
    \centering
    \includegraphics[width=0.9\textwidth]{FIG/ex10/rdf_comparison.png}
    \caption{$g(r)$ for all simulated $r_{cut}$ values}
    \label{ex10:rdf_comparison}
\end{figure}

As we can see for all the cutoff radii the $g(r)$ reveal that the system is in a liquid state, but for higher cutoff this determination is clearer.

\paragraph{NVT ensemble} In order to simulate the same LJ system in an NVT ensemble we need to introduce a thermostat in the simulation. \\
We do 2 separate runs: the first with velocity rescaling (non-canonical) and the second with the Nose-Hoover thermostat (canonical). For both runs we simulate the system at $T^* = 2$ for different decreasing number density values $\rho < 0.2$. \\
Given that LAMMPS outputs values per particle we should find in each simulation $\langle E_K \rangle = \frac{3}{2} T^* = 3$, while for the fluctuactions $$ \frac{\sigma^2_{T_K}}{\langle T_K \rangle^2} = \frac{2}{3}$$.
For a simulation with 50000 steps for equilibration and 450000 for production we found for the 2 runs at different density values, the following mean kinetic energy and mean temperature value, shown in table \ref{ex10:table}.


\begin{table}[ht]
    \centering
    \begin{tabular}{c|cc}
        \toprule
        \( \rho^* \) & \( \langle E_K \rangle \) & \( \langle T_K \rangle \) \\
        \midrule
        \multicolumn{3}{l}{\textbf{Non-Canonical}} \\
        0.075 & 2.94 & 1.98 \\
        0.100 & 2.95 & 1.98 \\
        0.125 & 2.95 & 1.98 \\
        0.150 & 2.96 & 1.98 \\
        0.175 & 2.96 & 1.98 \\
        \midrule
        \multicolumn{3}{l}{\textbf{Canonical}} \\
        0.075 & 2.49 & 1.68 \\
        0.100 & 2.90 & 1.96\\
        0.125 & 3.04 & 2.05\\
        0.150 & 3.16 & 2.11 \\
        0.175 & 2.81 & 1.88 \\
        \bottomrule
    \end{tabular}
    \caption{Table showing values for the mean temperature and kinetic energy for different values of number density for the non-canonical and canonical thermostat}
    \label{ex10:table}
\end{table}
