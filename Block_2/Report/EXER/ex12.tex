\chapter*{Exercise 12 : The multiple histogram method}
We are going to use in this exercise the multiple histogram method (MHM) to evaluate the energy variance per degree of freedom:
$$C_N(\beta) = \frac{1}{N} (\langle E^2 \rangle -\langle E \rangle^2) $$ 
For the grafted polymer model saw in chapter 6, where $N$ is the number of monomers. \\
Then we will find the maximum value of the inverse $\beta$ for which the maximum $C_N(\beta)$ is achieved, for different values of $N$. \\
Finally these value are plotted against $N^\phi$ where $\phi = 1/2$, to extrapolate the infinite size limit. \\

Before showing the results we comment briefly on the implementation of the MHM. 
In general in the MHM one has to put particular attention in evaluating the partition function, specifically in order to avoid over/under-flow due to the 
floating point structure, and specifically some special implementations are required, such as the log-sum-exp trick. \\
But due to the low number of degrees of freedom in our system we proceeded to a naive implementation of the algorithm, given that we are well within the floating point boundary. \\
Notice that anyway our algorithm is not scalable for large system, and a proper implementation is necessary in that case. \\
Also, to reduce our computation time, we raised the threshold to $10^{-3}$. \\
We present our results in figure \ref{ex12:infinite_size}.

\begin{figure}[htp]
    \centering
    \includegraphics[width=0.5\textwidth]{FIG/ex12/infinite_size.png}
    \caption{Infinite size limit determination}
    \label{ex12:infinite_size}
\end{figure}

As is clear, the evaluation is wrong, as it presents a negative temperature. This is due to the fact that the points are evaluated without error due to time constraint, so the whole procedure of best fit fails. \\
