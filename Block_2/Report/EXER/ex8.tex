\chapter*{Exercise 8: Off-lattice Monte Carlo simulations}

\subsection*{Pen \& Paper - Reduced units}
In order to convert from reduced units to SI units, specifically temperature and time, we simply do the conversion:

$$ T^* = \frac{k_B T}{\varepsilon} \iff T = \frac{\varepsilon T^*}{k_B} \qquad t^* = \frac{t}{\tau} \iff t = \tau t^* \quad, \tau = \sqrt{\frac{\bar{m}\sigma^2}{\varepsilon}} = \sigma\sqrt{\frac{\bar{m}}{k_B} \frac{k_B}{\varepsilon}} $$

Where $\sigma$, $\bar{m}$, $\varepsilon$ are respectively the reference unit for length, mass and energy for the system considered. \\

If we consider a reduced temperature $ T^* = 2 $, then for Argon and Krypton : 

$$ T_{\text{Ar}} = 239.6\ \text{K} \qquad  T_{\text{Kr}} = 328.0\ \text{K} $$

Moreover, if we consider a timestep of $\Delta t = 0.001\tau$, we have respectively for Argon and Krypton:

$$ \Delta t_{\text{Ar}} = 2.160\ \text{fs} \qquad  \Delta t_{\text{Kr}} = 2.650\ \text{fs} $$

\subsection*{Numerical exercise}

In this exercise we study a Lennard-Jones fluid using a Monte Carlo approach. \\
The Monte Carlo simulation we want to implement is simple: we want to perform a Monte Carlo sweep consisting of $N$ local moves, where $N = 500$ is the number of particles in the system, where each local move consist in a uniform displacement of a particle chosen at random. \\
Once a MC sweep is done, the new configuration is accepted or rejected using a Metropolis filter.
A total of $ T = 10000 $ sweeps are performed and then the pressure is measured, discarding an appropriate amount of sweeps for equilibration. \\
The displacement in the local move is chosen such that the acceptance rate of said filter is $\sim 50 \%$. \\

We begin by making some general considerations. Having set for our simulation the reference unit for length and energy repsctively $\sigma = 1$ and $\varepsilon = 1$, the reduced units coincides with the SI units. \\
One here has 2 choices in the code implementation: write a general code where each conversion between reduced and SI units is made, or to write a specific implementation for our system. \\
Both approaches have their merits : the first makes the code more mantainable and expandible, but slower, the second makes it more efficient, but problem-specific. \\
We choose for our implementation the second approach. \\

Of particular interest are some optimization made during the simulation :
\begin{itemize}
    \item Given that we want to evaluate quantities in function of the number density $\rho$, we can either change the number of particle $N$ or the simulation box volume $V$. \\ 
    While mathematically equivalent, given that most of the algorithms are dependent on $N$, is convenient to change $V$.
    \item In the implementation of the PBC one has to pay particular attention in not using the standard \fbox{\texttt{round}} implementation in Python, which implement the so called banker's round.\\
    One has instead to use the numpy function \fbox{\texttt{np.round}} (or even better \fbox{\texttt{np.rint}}) which implement the nearest integer round, the correct one and typical of C/Fortran.
    \item In the Metropolis filter if we pursue a naive approach, i.e. we evaluate the energy difference of the system between the final and the initial configuration as is, we found ourselves an algorithm of time 
    complexity $\mathcal{O}(N^2)$, where $N$ is the number of particle in the system. \\
    If we observe instead that in one local move the only quantity that changes is the potential contribution of the displaced particle, we can cut the time complexity to $\mathcal{O}(N)$. \\
    This also imply that the energy tail contribution in the context of the Metropolis filter is irrelevant.
\end{itemize}



