\chapter*{Exercise 13 : Exercise on the cell list}
We want to simulate in a square box of side $B=100$, $N$ Brownian particles with reflective boundary condition, with a mutual repulsive potential. \\
Half of the particle will have radius $R_1 = 1.25$ and the other half $R_2 = 1$ in order to prevent the system crystallization at $T^*=1$. \\
The initial positions will be initialized randomly inside the box and a non-compenetration check will be enforced through the evolution, that happens through the Euler-Maruyama algorithm.
The repulsive potential $U(r_{ij}) = \varepsilon \exp(-r_{ij}^2 / 2 \sigma_{ij})$ depends on the reference length $\sigma_{ij} = R_i + R_j$. We will cutoff the resulting force at $r_{cut} = 4\sigma_{ij}$. Moreover we set $k_B = 1$ and the diffusity through the mobility parameter $D_i = \mu_i T $ that depends on the particle radius $\mu_i = 1/R_i$. 
In order to achieve a large $N$ in our simulation, we will use the linked list cell method, in order to reduce the force calculation routine time complexity from $\mathcal{O}(N^2)$ to $\mathcal{O}(N)$. Given that $\max(r_{cut}) = 8R_1$ we choose this as side of our cell in the algorithm, so that is guaranteed that all force calculation are within the cutoff radius. \\
Now that we are set, in order to study the diffusivity we plot the mean square displacement along the x position of the particle of type 1 and 2, and of all the particles, for different values of $\rho = N/B^2$. 

\begin{figure}[ht]
    \centering
    \includegraphics[width=0.45\textwidth]{FIG/ex13/MSD_particle_number2.png} 
    \hspace{0.05\textwidth}
    \includegraphics[width=0.45\textwidth]{FIG/ex13/MSD_particle_number2_particle1.png}
    \caption{(Left) MSD for $\rho = 2\cdot 10^{-4}$. (Right) MSD for particles of type 1 and 2.}
    \label{fig:combined2}
\end{figure}

\begin{figure}[ht]
    \centering
    \includegraphics[width=0.45\textwidth]{FIG/ex13/MSD_particle_number10.png} 
    \hspace{0.05\textwidth}
    \includegraphics[width=0.45\textwidth]{FIG/ex13/MSD_particle_number10_particle1.png}
    \caption{(Left) MSD for $\rho = 1\cdot 10^{-3}$. (Right) MSD for particles of type 1 and 2}
    \label{fig:combined2}
\end{figure}

\begin{figure}[ht]
    \centering
    \includegraphics[width=0.45\textwidth]{FIG/ex13/MSD_particle_number50.png} 
    \hspace{0.05\textwidth}
    \includegraphics[width=0.45\textwidth]{FIG/ex13/MSD_particle_number50_particle1.png}
    \caption{(Left) MSD for $\rho = 5\cdot 10^{-3}$. (Right) MSD for particles of type 1 and 2.}
    \label{fig:combined3}
\end{figure}

\begin{figure}[ht]
    \centering
    \includegraphics[width=0.45\textwidth]{FIG/ex13/MSD_particle_number200.png} 
    \hspace{0.05\textwidth}
    \includegraphics[width=0.45\textwidth]{FIG/ex13/MSD_particle_number200_particle1.png}
    \caption{(Left) MSD for $\rho = 0.02$. (Right) MSD for particles of type 1 and 2.}
    \label{fig:combined2}
\end{figure}



We observe that the MSD follows $2D_i t$ up to a saturation due to the square confinement. For high $\rho$ we don't observe a caging in the MSD for lower times, 
but we observe it in the computation time, as the system makes at roughly half timestep begins to make many attempts ($\sim 10^6-10^{7}$) to try to pass to non-compenetration check. \\
In general we can see that the MSD for the 2 particles are roughly the same, while we can see that for particle 1 we have a slightly lower diffusivity.
