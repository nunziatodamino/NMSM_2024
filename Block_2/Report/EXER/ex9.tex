\chapter*{Exercise 9: Integration schemes}

The idea is to implement the velocity Verlet algorithm and the Gear predictor-correct algorithm of the 5th order specifically for the harmonic oscillator. \\
One can find said algorithms in the Algorithm appendix. \\
Before starting the quantitative analysis over these integration schemes, we start with a qualitative inspection against the analytical solution in figure \ref{ex9:schemes_comparison}. \\
Note that the integration timestep is $\Delta t = 0.001$ and $\omega = 0.1$.

\begin{figure}[H]
    \centering
    \includegraphics[width=0.9\textwidth]{FIG/ex9/integration_schemes_comparison.png}
    \caption{Comparison between the analytical solution for position and velocity in an harmonic oscillator with the velocit Verlet and the Gear 5th order predictor-corrector integration scheme.}
    \label{ex9:schemes_comparison}
\end{figure}

Is clear that the Gear predictor-correct algorithm of the 5th order presents for long integration time an effect of amplitude reduction and delay, while the velocity Verlet algorithm do not presents these problems. \\
Now we study the energy conservation of the 2 algorithms. Again we plot the total energy of the system over time for inspection in figure \ref{ex9:energy_comparison} and for each algorithm we plot $\frac{E_{algo}(t) - E_0}{E_0}$ respectively in figure \ref{fig:side_by_side} for the velocity Verlet algorithm and the Gear predictor-correct algorithm of the 5th order. 

\begin{figure}[H]
    \centering
    \includegraphics[width=0.9\textwidth]{FIG/ex9/energy_comparison.png}
    \caption{Comparison between the total energy in an harmonic oscillator and the energy evaluated with the velocity Verlet and the Gear 5th order predictor-corrector integration scheme.}
    \label{ex9:energy_comparison}
\end{figure}

\begin{figure}[H]
    \centering
    \includegraphics[width=0.45\textwidth]{FIG/ex9/verlet_energy_dev.png}
    \hspace{0.05\textwidth} % Adjust horizontal spacing between figures
    \includegraphics[width=0.45\textwidth]{FIG/ex9/gear_energy_dev.png}
    \caption{(Left) Relative energy deviation from the analytical value for the velocity Verlet integration scheme. (Right) Relative energy deviation from the analytical value for the Gear 5th order predictor-corrector integration scheme.}
    \label{fig:side_by_side}
\end{figure}

From this analysis we conclude, as we expected, that the velocity Verlet algorithm is symplectic while the Gear predictor-corrector is not. \\
Lastly we focus on the velocity Verlet to evaluate its stability for different values of $\omega$. We choose for each iteration a timestep $\Delta t= 0.01\omega^{-1}$ and obtain figures \ref{fig:stability_comparison}.

\begin{figure}[h!]
    \centering
    % First row: Two images side by side
    \begin{minipage}{0.45\textwidth}
        \centering
        \includegraphics[width=\textwidth]{FIG/ex9/stability_comparison_0.png}
        \caption{Position and velocity deviation for the velocity Verlet integration scheme for $\omega = 0.01$.}
        \label{fig:stability_comparison_0}
    \end{minipage}
    \hfill
    \begin{minipage}{0.45\textwidth}
        \centering
        \includegraphics[width=\textwidth]{FIG/ex9/stability_comparison_1.png}
        \caption{Position and velocity deviation for the velocity Verlet integration scheme for $\omega=1$.}
        \label{fig:stability_comparison_1}
    \end{minipage}
    
    % Second row: Single image below
    \vspace{1em}
    \begin{minipage}{0.45\textwidth}
        \centering
        \includegraphics[width=\textwidth]{FIG/ex9/stability_comparison_2.png}
        \caption{Position and velocity deviation for the velocity Verlet integration scheme for $\omega=100$.}
        \label{fig:stability_comparison_2}
    \end{minipage}
    
    \caption{Comparison of position and velocity deviations for the velocity Verlet integration scheme.}
    \label{fig:stability_comparison}
\end{figure}

We observe that the deviation is smaller for small $\omega$.